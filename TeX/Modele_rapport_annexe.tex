%%%%%%%%%%%%%%%%%%%%%%%%%%%%%
    %%  Modèle rapport         %%
    %%  Annexe                 %%
    %%  Version 2015.06.28     %%
    %%%%%%%%%%%%%%%%%%%%%%%%%%%%%


%%%%%%%%%%%%%%%%%%%%%%%%%%%%%%%%%%%%% Châpeau
\clearpage
\appendixpage
\addappheadtotoc
\ifbool{rectoVerso}{\fancyhead[LE,RO]{\nomAnnexe}}{\fancyhead[RE,RO]{\nomAnnexe}}
%%%%%%%%%%%%%%%%%%%%%%%%%%%%%%%%%%%%% Configurations post-châpeau
\titleformat{\subsection}[hang]{\large\bf}{\color{couleurSubsection}\Alph{subsection}}{5mm}{\color{couleurSubsection}} %% Pour le titre
\renewcommand{\thesubsection}{\textcolor{couleurSubsection}{\Alph{subsection}}} %% Pour la table des matières
\ifbool{compteurFigTabAvecSec}{ %% Pour les compteurs de figures et de tableaux
	\numberwithin{figure}{subsection} %% Ajouter le numéro de la subsection pour les figures
	%\renewcommand{\thefigure}{\Alph{subsection}.\arabic{figure}} %% Changement du style si nécessaire
	\numberwithin{table}{subsection} %% Ajouter le numéro de la subsection pour les tableaux
	%\renewcommand{\thetable}{\Alph{subsection}.\arabic{table}} %% Changement du style si nécessaire
}{}
%%%%%%%%%%%%%%%%%%%%%%%%%%%%%%%%%%%%% Texte


%% A %%
\subsection{Calcul bourrin}

    \begin{table}[!htb]
        \centering \begin{tabular}{|ccc|}
            \hline
            A & B & C \\
            \hline
            data & data & data \\
            data & data & data \\
            data & data & data \\
            \hline
        \end{tabular}
        \caption{Légende}
        \label{etiquette3}
    \end{table}

    Voici des exemples:
    \begin{itemize}
        \item ex1;
        \item ex2.
    \end{itemize}

    \vspace*{1ex}
    Paragraphe.
    
    